\documentclass[11pt]{beamer}

% beamer posterの利用に関する設定
% A0 に合わせて 4 : 3 に
\usepackage[orientation=landscape,size=custom,width=112,height=84,scale=2.0]{beamerposter}

% メインフォントに関する設定
\usepackage{fontspec}
\setmainfont[Ligatures=TeX,BoldFont=Noto Sans Bold]{Noto Sans}

% CJKに対応させる
% 太字フォントを指定しないと疑似ボールドになる
\usepackage{xeCJK}
\setCJKmainfont[BoldFont=Noto Sans CJK JP Bold]{Noto Sans CJK JP}

% 追加のフォント
\newfontfamily{\notosansmono}{Noto Sans Mono}

% tikz
\usepackage{tikz}
\usepackage{pgfplots}
\usepackage{bchart}
\usetikzlibrary{positioning}
\usetikzlibrary{arrows.meta}

% beamerのテーマ・スタイルに関する設定
\setbeamertemplate{navigation symbols}{} % 右下のナビゲーションアイコンに消えてもらう
\setbeamercolor{normal text}{fg=black,bg=white}
\setbeamercolor{structure}{fg=red}

% blockのスタイル
\usepackage[most]{tcolorbox} % mostがないとenhanced(tikzを用いる装飾表現)が使えない
\setbeamertemplate{block begin}{
	\begin{tcolorbox}[
				enhanced,
				fonttitle=\bfseries,
				colframe=gray,
				colback=gray!5!,
				colbacktitle=gray!80,
				attach boxed title to top left={yshift=-2mm, xshift=2mm},
				title=\insertblocktitle
			]
		\vskip2mm
}
\setbeamertemplate{block end}{
	\end{tcolorbox}
}

% href
\usepackage{hyperref}
\hypersetup{
	colorlinks=true,
	linkcolor=blue,
	filecolor=magenta,
	urlcolor=cyan,
	pdfnewwindow=true}

% その他
\usepackage{fancybox}
\usepackage{here}

\begin{document}
\begin{frame}[t]
	\vskip0.5em
	\begin{columns}[t]
		\begin{column}{.10\linewidth}
			\raggedleft
			\Large
			P5-18
		\end{column}

		\begin{column}{.80\linewidth}
			\centering
			{\Large ニューラル文法誤り訂正のための多様な規則を用いる人工誤り生成}
			\vskip0.3em

			$\underset{
				\hspace{-100em}
				\text{\small (東工大\ 岡崎研\ 修士1年)}
				\hspace{-100em}
			}{
				\underline{\text{古山翔太}}
			}$
			\hskip2.5em 高村大也
			\hskip2.5em 岡崎直観
			\hskip2em (東工大/産総研)
		\end{column}

		\begin{column}{.10\linewidth}
			\raggedright
			\vspace*{-3cm}
			\begin{figure}[H]
				\hspace*{-16cm}
				\vspace*{-11cm}
				\includegraphics[width=6cm]{logo.png}
			\end{figure}
			\vspace*{6cm}
			\begin{figure}[H]
				\hspace*{-2cm}
				\vspace*{4cm}
				\includegraphics[width=10cm]{banneraist.png}
			\end{figure}
		\end{column}
	\end{columns}

	\vspace*{-0.5em}

	\begin{columns}[t]
		\begin{column}{.33\linewidth}

			\vspace*{-0.3em}
			\begin{figure}[H]
				\centering
				\begin{tikzpicture}[
							font=\sffamily,
							scale=1.0,
							transform shape]
						\node[
								rectangle,
								text width=180,
								text height=50,
								fill=blue!20,
								draw=black,
								outer sep=1mm
							] at (0, 0) {};
						\node at (0, 0) {\textbf{概要}};
				\end{tikzpicture}
			\end{figure}
			\vspace*{-0.8em}

			\begin{tcolorbox}[
				enhanced,
				fonttitle=\bfseries,
				colframe=gray,
				colback=gray!5!,
				colbacktitle=gray!80,
				attach boxed title to top left={yshift=-2mm, xshift=2mm},
			]
				\begin{itemize}
					\item 文法誤り訂正の事前学習のデータをルールベースで人工的に生成
					\item 多様なルールの組合せが効果的
				\end{itemize}
			\end{tcolorbox}
			\vspace*{-0.5em}

			\begin{figure}[H]
				\centering
				\begin{tikzpicture}[
							font=\sffamily,
							scale=1.0,
							transform shape]
						\node[
								rectangle,
								text width=180,
								text height=50,
								fill=blue!20,
								draw=black,
								outer sep=1mm
							] at (0, 0) {};
						\node at (0, 0) {\textbf{背景}};
				\end{tikzpicture}
			\end{figure}
			\vspace*{-0.8em}

			\begin{block}{
					\begin{tabular}{l}
						ニューラル文法誤り訂正のパラダイム \hspace{2em} \\
						\multicolumn{1}{r}{
							「事前学習+再学習」
						}
					\end{tabular}}
				\begin{figure}[h]
					\centering
					\begin{tikzpicture}[
							>=latex,
							font=\sffamily,
							scale=1.0,
							transform shape]
						\node[
								rectangle,
								text width=180,
								text height=120,
								rounded corners=20mm,
								draw=green,
								fill=green!20,
								outer sep=5mm
							] (data1) at (0, 8) {};
						\node[
								rectangle,
								text width=150,
								text height=100,
								rounded corners=20mm,
								draw=green,
								fill=green!20,
								outer sep=5mm
						] (data2) at (12, 8) {};
						\node at (0, 9) {\large 人工};
						\node at (0, 7) {\large データ};
						\node at (12, 9) {\small 学習者};
						\node at (12, 7) {\small コーパス};
						\draw[-{Triangle[width=16pt,length=8pt]}, line width=2pt] (-6, 8) -- (-4, 8);
						\draw[-{Triangle[width=16pt,length=8pt]}, line width=2pt] (18, 8) -- (16, 8);
						\node at (-8.5, 8.75) {\small 大規模};
						\node at (-8.5, 7.25) {\small に作る};
						\node at (20.5, 8.75) {\small 増やす};
						\node at (20.5, 7.25) {\small の大変};
						\node[
								rectangle,
								text width=100,
								text height=80,
								rounded corners,
								draw=blue,
								fill=blue!20,
								outer sep=1mm
							] (pretrain) at (0, 0) {};
						\node[
								rectangle,
								text width=100,
								text height=80,
								rounded corners,
								draw=blue,
								fill=blue!20,
								outer sep=1mm
							] (finetune) at (12, 0) {};
						\draw[-{Triangle[width=30pt,length=16pt]}, line width=2pt] (0, 5.3) -- (pretrain);
						\draw[-{Triangle[width=30pt,length=16pt]}, line width=2pt] (12, 5.3) -- (finetune);
						\draw[-{Triangle[width=30pt,length=16pt]}, line width=2pt] (pretrain) -- (finetune);
						\node[anchor=east] at (-0.3, 4) {事前学習};
						\node[anchor=west] at (12.3, 4) {再学習};
						\node at (6, -1.0) {\small 事前学習};
						\node at (6, -2.5) {\small 済みモデル};
						\node at (0, +0.7) {\small Trans-};
						\node at (0, -0.7) {\small former};
						\node at (12, +0.7) {\small Trans-};
						\node at (12, -0.7) {\small former};
					\end{tikzpicture}
				\end{figure}
				大規模な \textbf{人工データ} による\textbf{事前学習}
				\\
				\hspace*{0.5em} + \textbf{学習者コーパス} による\textbf{再学習}が主流
				\vskip0.2em
				\begin{itemize}
					\item ``人工データ''をどのように作るかが重要
				\end{itemize}
			\end{block}

			\begin{block}{人工誤り生成とは? \ \ その課題は?}
				\begin{figure}[H]
					\centering
					\begin{tikzpicture}[
							font=\sffamily,
							scale=1.0,
							transform shape]
						\node (x) at (0, 6) {I went \textcolor{blue}{to} Tokyo.};
						\node (y) at (22, 2.8) {I went \textcolor{red}{for} Tokyo.};
						\node[
								rectangle,
								text width=240,
								text height=70,
								rounded corners,
								draw=red,
								fill=red!20,
								outer sep=1mm
							] (s) at (8, 3) {};
						\node at (8, 3.8) {\small 誤り生成規則};
						\node at (8, 2.2) {\small ``to $\to$ for''};
						\draw[-{Triangle[width=30pt,length=16pt]}, line width=2pt] (x.south) -- (0, 3) -- (s.west);
						\draw[-{Triangle[width=30pt,length=16pt]}, line width=2pt] (s.east) -- (15.3, 3);
						\draw[-{Triangle[width=30pt,length=16pt]}, line width=2pt, dashed] (17, 4.0) -- (17, 6) -- (x.east);
						\node at (22.0, 6.5) {\small 訂正モデルは};
						\node at (22.0, 5.0) {\small 逆を学習};
					\end{tikzpicture}
				\end{figure}
				\vskip0.2em
				文法誤り訂正の学習データを作るために,人工的に誤りデータを生成すること
				\vskip0.2em
				\begin{itemize}
					\item 本研究ではルールベースで誤り生成
						\begin{itemize}
							\item \normalsize 少ないルールでは誤り文が偏る...
						\end{itemize}
				\end{itemize}
				\begin{itemize}
					\item 多くのルールを用いて良い誤り文を偏りなく,多様に生成したい!
					\item 人工誤りデータの質が向上し,文法誤り訂正の性能向上が期待
				\end{itemize}
				\vspace*{0.3em}
			\end{block}

		\end{column}

		\begin{column}{.33\linewidth}

			\vspace*{-0.3em}
			\begin{figure}[H]
				\centering
				\begin{tikzpicture}[
							font=\sffamily,
							scale=1.0,
							transform shape]
						\node[
								rectangle,
								text width=180,
								text height=50,
								fill=blue!20,
								draw=black,
								outer sep=1mm
							] at (0, 0) {};
						\node at (0, 0) {\textbf{手法}};
				\end{tikzpicture}
			\end{figure}
			\vspace*{-0.8em}

			\begin{block}{どのように多様な誤り文を生成するか?}
				\begin{figure}[h]
					\centering
					\begin{tikzpicture}[
							font=\sffamily,
							scale=1.0,
							transform shape]
						\node[
								rectangle,
								text width=950,
								text height=40,
								rounded corners,
								draw=blue,
								fill=blue!20,
								outer sep=3mm
							] (s) at (16, 0) {};
						\node[anchor=west] at (-1, 0) {\footnotesize 修正文としての文法的な文};
						\foreach \x / \y in {0/ I, 1/ went, 2/ to, 3/Tokyo, 4/.}{
							\node at (\x * 4 + 16, 0) {\small \y};}
						\draw[-{Triangle[width=48pt,length=16pt]}, line width=16pt] (16, -1.5) -- (16, -2.5);

						\node[
								rectangle,
								text width=950,
								text height=160,
								rounded corners,
								draw=blue,
								fill=blue!20,
								outer sep=3mm
							] (s) at (16, -6.0) {};
						\node[anchor=west] at (-1.5, -3.3) {\small ルール1};
						\node[anchor=west] at (-1.0, -4.6) {\footnotesize  「前置詞 to」};
						\node[
								rectangle,
								text width=260,
								text height=80,
								rounded corners,
								draw=green,
								fill=green!20,
								outer sep=3mm
							] (s) at (4.5, -7.3) {};
						\node at (2, -5.8) {\scriptsize ベータ分布};
						\node at (2.2, -7.2) {\scriptsize $\mu$ = 0.05};
						\node at (2.2, -8.2) {\scriptsize $\sigma$ = 0.05};
						\node at (5.2, -7.7) {\scriptsize $\rightsquigarrow$};
						\node at (7.5, -6.5) {\scriptsize 閾値};
						\node at (7.5, -7.7) {\scriptsize 0.2};
						\node[
								rectangle,
								text width=12,
								text height=8,
								rounded corners,
								draw=green,
								fill=green!20,
								outer sep=0mm
							] (s) at (24.15, -4.0) {};
						\foreach \x / \y in {0/ I, 1/ went, 2/ to, 3/Tokyo, 4/.}{
							\node at (\x * 4 + 16, -4.0) {\small \y};}
						\node at (25, -3.2) {\tiny \textcolor{red}{OK}};
						\draw[-{Triangle[width=16pt,length=16pt]}, line width=4pt] (24.15, -4.5) -- (24.15, -5.5);
						\draw[-{Triangle[width=16pt,length=16pt]}, line width=4pt] (24.15, -6.5) -- (24.15, -7.5);
						\draw[-{Triangle[width=8pt,length=16pt]}, line width=1pt] (8.5, -7.7) -- (20.15, -6.0);
						\node at (21.15, -6.0) {\footnotesize 0.2};
						\node at (22.65, -6.0) {\footnotesize $>$};
						\node at (24.15, -6.0) {\footnotesize 0.1};
						\draw[-{Triangle[width=16pt,length=16pt]}, line width=4pt] (27, -6.0) -- (25, -6.0);
						\node at (30, -5.5) {\scriptsize 一様分布から};
						\node at (30, -6.5) {\scriptsize サンプル};
						\foreach \x / \y in {0/ I, 1/ went, 2/ \textcolor{red}{for}, 3/Tokyo, 4/.}{
							\node at (\x * 4 + 16, -8.0) {\small \y};}
						\draw[-{Triangle[width=48pt,length=16pt]}, line width=16pt] (16, -9.5) -- (16, -10.5);

						\node[
								rectangle,
								text width=950,
								text height=160,
								rounded corners,
								draw=blue,
								fill=blue!20,
								outer sep=3mm
							] (s) at (16, -14.0) {};
						\node[anchor=west] at (-1.5, -11.3) {\small ルール2};
						\node[anchor=west] at (-1.0, -12.6) {\footnotesize 「1人称単数主格代名詞」};
						\node[
								rectangle,
								text width=260,
								text height=80,
								rounded corners,
								draw=green,
								fill=green!20,
								outer sep=3mm
							] (s) at (4.5, -15.3) {};
						\node at (2, -13.8) {\scriptsize ベータ分布};
						\node at (2.2, -15.2) {\scriptsize $\mu$ = 0.03};
						\node at (2.2, -16.2) {\scriptsize $\sigma$ = 0.03};
						\node at (5.2, -15.7) {\scriptsize $\rightsquigarrow$};
						\node at (7.5, -14.5) {\scriptsize 閾値};
						\node at (7.5, -15.7) {\scriptsize 0.05};
						\node[
								rectangle,
								text width=12,
								text height=8,
								rounded corners,
								draw=green,
								fill=green!20,
								outer sep=0mm
							] (s) at (16.15, -12.0) {};
						\foreach \x / \y in {0/ I, 1/ went, 2/ for, 3/Tokyo, 4/.}{
							\node at (\x * 4 + 16, -12.0) {\small \y};}
						\node at (17, -11.2) {\tiny \textcolor{red}{OK}};
						\draw[-{Triangle[width=16pt,length=16pt]}, line width=4pt] (16.15, -12.5) -- (16.15, -13.5);
						\draw[-{Triangle[width=8pt,length=16pt]}, line width=1pt] (8.5, -15.7) -- (12.15, -15.0);
						\node at (13.15, -15.0) {\footnotesize 0.05};
						\node[rotate=20] at (14.65, -14.5) {\footnotesize $\not >$};
						\node at (16.15, -14.0) {\footnotesize 0.9};
						\foreach \x / \y in {0/ I, 1/ went, 2/ for, 3/Tokyo, 4/.}{
							\node at (\x * 4 + 16, -16.0) {\small \y};}
						\draw[-{Triangle[width=48pt,length=16pt]}, line width=16pt] (16, -17.5) -- (16, -18.5);

						\node[
								rectangle,
								text width=950,
								text height=40,
								rounded corners,
								draw=red,
								fill=red!20,
								outer sep=3mm
							] (s) at (16, -20) {};
						\node[anchor=west] at (2, -20) {\small 人工誤り文};
						\foreach \x / \y in {0/ I, 1/ went, 2/ for, 3/Tokyo, 4/.}{
							\node at (\x * 4 + 16, -20) {\small \y};}
						
					\end{tikzpicture}
				\end{figure}
				\vspace*{-0.2em}
				ルールを順次,独立に適用 \\
				文ごとに誤り確率を変え多様な誤り文を生成 \\
				誤り生成規則は全\textbf{5}カテゴリ・\textbf{188}種類 \textcolor{blue}{${}^a$} \\
				\hspace*{2em} {\small (機能語・活用・語順・単語選択・表記体系)}
			\end{block}

			\vspace*{-0.5em}
			\begin{figure}[H]
				\centering
				\begin{tikzpicture}[
							font=\sffamily,
							scale=1.0,
							transform shape]
						\node[
								rectangle,
								text width=180,
								text height=50,
								draw=black,
								fill=blue!20,
								outer sep=1mm
							] at (0, 0) {};
						\node at (0, 0) {\textbf{結果}};
				\end{tikzpicture}
			\end{figure}

			\vspace*{-0.8em}
			\begin{tcolorbox}[
					enhanced,
					fonttitle=\bfseries,
				colframe=gray,
				colback=gray!5!,
				colbacktitle=gray!80,
					attach boxed title to top left={yshift=-2mm, xshift=2mm},
				]
				モデル: Transformer big \\
				単言語コーパス: 1億2400万文 {\scriptsize (WMT ニュースタスク)} \\
				学習者コーパス: 62万文
				\begin{table}[H]
					\centering
					\small
					\tabcolsep 6pt
					\label{tab:result}
					\begin{tabular}{lcccc}
						\hline
						& \multicolumn{2}{c}{\footnotesize
								\hspace{-0em}{$\def\arraystretch{0.5}\begin{array}{cc}\vspace{-0.5em}\\\multicolumn{2}{c}{\text{BEA-19}}\\\text{valid} & \text{test}\\\end{array}$}\hspace{-2em}}
						& \hspace{-2em}{\footnotesize $\def\arraystretch{0.5}\begin{array}{c}\vspace{-0.5em}\\\text{CoNLL}\\\text{14}\\\end{array}$}\hspace{-2em}
						& {\footnotesize JFLEG}
						\\
						& {\scriptsize (F${}_{0.5}$)}
						& {\scriptsize (F${}_{0.5}$)}
						& {\scriptsize (F${}_{0.5}$)}
						& {\scriptsize (GLEU)}
						\\
						\hline
						ベースライン
						& 44.73 & - & 53.88 & 57.37 \\
						\hline
						事前学習
						& 41.79 & - & 54.92 & 58.35\\
						\ \ +再学習
						& 52.83 & - & 63.25 & 62.23 \\
						\ \ \ {\footnotesize +アンサンブル+リスコア}
						& 54.68 & 72.51 & \underline{65.43} \makebox[0pt]{\textcolor{blue}{${}^b$}}\ignorespaces & \underline{63.69} \\
						{$\def\arraystretch{0.5}\hspace*{-2.5em}\begin{array}{c}\vspace*{-2em}\\\text{\tiny +再学習 +ドメイン適応}\\\text{\tiny \hspace*{10em} +アンサンブル+リスコア}\\\vspace*{-2em}\\\end{array}$}
						& 56.49 & \underline{72.76} & - & - \\
						\hline
						{\footnotesize Grundkiewicz+ 19} {\tiny (ルールベース誤り生成)}
						& 53.00 & 69.47 & 64.16 & 61.22 \\
						{\footnotesize Omelianchuk+ 20} {\tiny (タグ付けモデル)}
						& - & \textbf{73.7} & 66.5 & - \\
						{\footnotesize Lichtarge+ 20} {\tiny (折返し翻訳の誤り生成)}
						& - & 73.0 & \textbf{66.8} & \textbf{64.9} \\
						\hline
					\end{tabular}
				\end{table}
				既存のルールベース誤り生成よりは良い\\
				タグ付モデル・MTベース誤り生成に及ばず
			\end{tcolorbox}
		\end{column}

		\begin{column}{.33\linewidth}

			\vspace*{-0.3em}
			\begin{figure}[H]
				\centering
				\begin{tikzpicture}[
							font=\sffamily,
							scale=1.0,
							transform shape]
						\node[
								rectangle,
								text width=180,
								text height=50,
								fill=blue!20,
								draw=black,
								outer sep=1mm
							] at (0, 0) {};
						\node at (0, 0) {\textbf{議論}};
				\end{tikzpicture}
			\end{figure}

			\vspace*{-0.8em}
			\begin{block}{誤り生成規則の多様さは重要か?}
				\begin{figure}[H]
					\centering
					{\footnotesize 事前学習 (8M文 $\times$ 10エポック) + 再学習} \\
					\footnotesize
					\begin{tikzpicture}[scale=1.0,transform shape]
						\begin{axis}[
								ymin=0,
								ymax=1,
								xmin=40,
								xmax=50,
								width=27cm,
								height=20cm,
								xbar=0pt,
								/pgf/bar shift=0pt,
								xlabel near ticks, xlabel shift={-8pt},
								ylabel near ticks, ylabel shift={-2pt},
								bar width=1.0cm,
								xlabel={BEA-19 valid (F${}_{0.5}$)},
								ytickmin=全カテゴリ,
								ytickmax=ベースライン,
								xtick={40,41,42,43,44,45,46,47,48,49,50},
								symbolic y coords={0,全カテゴリ,全\ -\ 表記体系,全\ -\ 単語選択,全\ -\ 語順,全\ -\ 活用,全\ -\ 機能語,+\ 表記体系,+\ 単語選択,+\ 語順,+\ 活用,+\ 機能語,ベースライン,1},
								nodes near coords,
								every node near coord/.append style={rotate=0, anchor=west, xshift=0.0cm, font=\scriptsize},
								yticklabel style={anchor=west,xshift=-245pt},
							]
							\addplot[xbar, draw=red, fill=red!50] coordinates {
								(44.73,ベースライン)};
							\addplot[xbar, draw=magenta, fill=magenta!50, bar width=1.0cm] coordinates {
									(45.12,+\ 機能語)
									(44.96,+\ 活用)
									(44.42,+\ 語順)
									(44.86,+\ 単語選択)
									(46.55,+\ 表記体系)};
							\addplot[xbar, draw=cyan, fill=cyan!50, bar width=1.0cm] coordinates {
									(48.72,全\ -\ 機能語)
									(48.45,全\ -\ 活用)
									(48.60,全\ -\ 語順)
									(48.56,全\ -\ 単語選択)
									(47.51,全\ -\ 表記体系)};
							\addplot[xbar, draw=blue, fill=blue!50, bar width=1.0cm] coordinates {
									(48.59,全カテゴリ)};
							\addplot[red, line legend, sharp plot, line width=2pt, nodes near coords={}]
								coordinates {(44.73,0) (44.73,1)};
							\addplot[blue, line legend, sharp plot, line width=2pt, nodes near coords={}]
								coordinates {(48.59,0) (48.59,1)};
							\node[rotate=-90] at (axis cs: 44.5,+\ 表記体系) {\scriptsize ベ};
							\node[rotate=-90] at (axis cs: 44.5,全\ -\ 機能語) {\scriptsize ー};
							\node[rotate=-90] at (axis cs: 44.5,全\ -\ 活用) {\scriptsize ス};
							\node[rotate=-90] at (axis cs: 44.5,全\ -\ 語順) {\scriptsize ラ};
							\node[rotate=-90] at (axis cs: 44.5,全\ -\ 単語選択) {\scriptsize イ};
							\node[rotate=-90] at (axis cs: 44.5,全\ -\ 表記体系) {\scriptsize ン};
							\node[rotate=-90] at (axis cs: 48.3,+\ 活用) {\scriptsize 全カテゴリ};
						\end{axis}
					\end{tikzpicture}
				\end{figure}
				\vspace*{-0.5em}
				1カテゴリのみの誤り生成は効果が薄い \\
				1カテゴリを除いても,性能低下しにくい
				\begin{itemize}
					\item 多様な誤り生成の統合は効果的
				\end{itemize}
			\end{block}

			\begin{block}{単言語コーパスの規模は重要か?}
				\begin{figure}[H]
						\centering
						\small
						\vspace*{-0.1em}
						\begin{tikzpicture}[scale=1.0,transform shape]
							\begin{axis}[
									ytick scale label code/.code={},
									xmin=0,
									ymin=46,
									ymax=52,
									xtick=data,
									xticklabels={4M,8M,16M,32M},
									width=28cm,
									height=14cm,
									grid=major,
									ytick={46,48,50,52},
									ylabel={\footnotesize \hspace*{-2em} BEA-19 valid {\scriptsize ($\text{F}_{0.5}$)}\hspace*{0em}},
									xlabel near ticks, xlabel shift={-2pt},
									ylabel near ticks, ylabel shift={-2pt},
									legend style={
										legend image post style={solid},
										legend pos=south east,
								}]
							\addplot+[line width=3pt, mark size=6pt] coordinates {(4, 46.37) (8, 48.59) (16, 50.74) (32, 51.76)};
							\addplot+[line width=3pt, mark size=6pt] coordinates {(4, 50.48) (8, 50.74) (16, 50.74) (32, 50.81)};
							\legend{エポック数を固定した場合,ステップ数を固定した場合} 
							\node at (axis cs: 3.0, 46.87) {\footnotesize 10};
							\node at (axis cs: 7.0, 49.09) {\footnotesize 10};
							\node at (axis cs: 15.0, 51.24) {\footnotesize 10};
							\node at (axis cs: 33.3, 51.56) {\footnotesize 10};
							\node at (axis cs: 3.0, 50.98) {\footnotesize 40};
							\node at (axis cs: 7.0, 51.24) {\footnotesize 20};
							\node at (axis cs: 33.0, 50.51) {\footnotesize 5};
						\end{axis}
						\end{tikzpicture}
					\vskip0.0em
					{\small 単言語コーパスの規模}
				\end{figure}
				\vspace*{0.2em}
				単言語コーパスを4M,8M,16M,32Mと変え \\
				エポック数固定(\textcolor{blue}{10})と\textcolor{red}{40→20→10→5}で比較
				\vspace*{0.3em}
				\begin{itemize}
					\item 単言語コーパスの規模が大きければいいというわけではない
					\item 低資源の言語にも良いルールを用いれば高性能なモデルを構築可能?
				\end{itemize}
			\end{block}

		\end{column}
	\end{columns}
	\vskip0.3em
	\textcolor{blue}{${}^a$} 誤り生成規則は実験に用いたソースコードとともに公開しています.(\href{https://github.com/nymwa/arteraro}{github.com/nymwa/arteraro}) \\
	\textcolor{blue}{${}^b$} CoNLL-14に関する論文中の値に誤りがありましたので,当ポスター,修正原稿 (\href{https://github.com/nymwa/anlp\_nlp2021\_aeg/blob/main/nlp2021.pdf}{リンク})では訂正した値を掲載しました.
\end{frame}
\end{document}

